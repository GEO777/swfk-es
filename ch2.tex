% ch2.tex
% This work is licensed under the Creative Commons Attribution-Noncommercial-Share Alike 3.0 New Zealand License.
% To view a copy of this license, visit http://creativecommons.org/licenses/by-nc-sa/3.0/nz
% or send a letter to Creative Commons, 171 Second Street, Suite 300, San Francisco, California, 94105, USA.


\chapter{8 multiplicado por 3.57 igual a$\ldots$}\label{ch:8multipliedby3.57}

¿Cuánto es 8 multiplicado por 3.57? Tendrías que utilizar una calculadora, ¿a que sí? Bueno, tal vez eres superlisto y puedes calcular con decimales mentalmente---bueno, no es importante.  Tambien puedes hacer el cálculo con la consola de Python. Inicia la consola de nuevo (mira el Capítulo~\ref{ch:notallsnakeswillsquishyou} para más información, si te lo hubieras saltado por alguna extraña razón), y cuando veas el prompt en pantalla, teclea 8$*$3.57 y pulsa la tecla Intro:

\begin{listing}
\begin{verbatim}
Python 3.0 (r30:67503, Dec  6 2008, 23:22:48) 
Type "help", "copyright", "credits" or "license" for more information.
>>> 8 * 3.57
28.559999999999999
\end{verbatim}
\end{listing}

La estrella (*), o tecla de asterico, se utiliza como símbolo de multiplicación\index{multiplication}, en lugar del simbolo de multiplicar tradicional (\textsf{X}) que utilizas en el cole (es necesario que utilices la tecla de estrell porque el ordenador no tendría forma de distinguir si lo querías era teclear la letra \emph{x} o el símbolo de la multiplicación \textsf{X}).  ¿Qué te parece si probamos un cálculo que sea un poco más útil?

Supón que haces tus obligaciones de la casa una vez a la semana, y por ello te dan 5 euros, y que haces de chico repartidor de periódicos 5 veces por semana cobrando 30 euros---¿Cuánto dinero podrías ganar en un año?

\begin{figure}[t]
\begin{center}
\fbox{\colorbox{PaleBlue}{\parbox{.75\linewidth} {
\subsection*{¿!¿!Python está roto!?!?}

Si has probado a calcular 8 x 3.57 con una calculadora el resultado que se muestra será el siguiente:\\

\textsf{
28.56\\
}

\noindent
¿Porqué el resultado de Python es diferente? ¿Está roto?\\

\noindent
En realidad no. La razón de esta diferencia está en la forma en que se hacen los cálculos en el ordenador con los números de coma flotante\index{floating point} (Los números que tienen una coma y decimales).  Es algo complejo y que confunde a los principianes, por lo que es mejor que recuerdes únicamente que cuando estás trabajando con decimales, \emph{a veces} el resultado no será exactamente el experado. Esto puede pasar con las multiplicaciones, divisiones, sumas y restas.
}}}
\end{center}
\end{figure}

Si lo escribieramos en un papel, lo calcularíamos como sigue:
\begin{verbatim}
(5 + 30) x 52
\end{verbatim}

Es decir, 5 + 30 euros a la semana multiplicado por las 52 semanas del año.  \begin{samepage} Desde luego, somos chicos listos por lo que sabemos que 5 + 30 es 35, por lo que la fórmula en realidad es:

\begin{verbatim}
35 x 52
\end{verbatim}
\end{samepage}

Que es suficientemente sencilla como para hacerla con una calculadora o a mano, con lápiz y papel. Pero tambié podemos hacer todos los cálculos con la consola:

\begin{listing}
\begin{verbatim}
>>> (5 + 30) * 52
1820
>>> 35 * 52
1820
\end{verbatim}
\end{listing}

Veamos, ¿Qué pasaría si te gastases 10 euros cada semana? ¿Cuánto te quedaría a final de año? En papel podemos escribir este cálculo de varias formas, pero vamos a teclearlo en la consola de Python: 

\begin{listing}
\begin{verbatim}
>>> (5 + 30 - 10) * 52
1300
\end{verbatim}
\end{listing}

Son 5 más 30 euros menos 10 euros multiplicado por las 52 semanas que tiene el año. Te quedarán 1300 euros a final de año. Estupendo, pero aún no es quesea muy útil lo que hemos visto. Todo se podría haber hecho con una calculadora. Pero volveremos con este asunto más tarde para mostrarte como hacer que sea mucho más útil.  

En la consola de Python puedes multiplicar\index{multiplication} y sumar\index{addition} (obvio), y restar\index{subtraction} y dividir\index{division}, además de un gran puñado de ootras operaciones matemáticas que no vamos a comentar ahora. Por el momento nos quedamos con los símbolos básicos de matemáticas que se pueden usar en Python (en realidad, en este lenguaje, se les llama operadores\index{operators}):

\begin{center}
\begin{tabular}{|c|c|}
\hline
+ & Suma \\
\hline
- & Resta \\
\hline
* & Multiplicación \\
\hline
/ & División \\
\hline
\end{tabular}
\end{center}

La razón por la que se usa la barra inclinada hacia adelante (/) para la división, es que sería muy difícil dibujar la línea de la división como se supone que lo haces en las fórmulas escritas (además de que ni se molestan en poner un símbolo de división $\div$ en el teclado del ordenador). Por ejemplo, si tuvieras 100 huevos y 20 cajas, podrías querer saber cuantos huevos deberías meter en cada caja, por lo que tendrías que dividr 100 entre 20, para lo que escribirías la siguiente fórmula:

\begin{displaymath}
\frac{100}{20}
\end{displaymath}

O si supieras como se escribe la división larga, lo harías así:

\begin{displaymath}
\longdiv{100}{20}
\end{displaymath}

O incluso lo podrías escribir de esta otra forma:

\begin{displaymath}
100 \div 20
\end{displaymath}

Sin embargo, en Python tienes que escribirlo así ``100 / 20''.

\emph{Lo que es mucho más sencillo, o eso me parece a mi.  Pero bueno, yo soy un libro---¿Qué voy a saber yo?}

\section{El uso de los paréntesis y el ``Orden de las Operaciones''}\index{order of operations}

En los lenguajes de programación se utilizan los paréntesis para controlar lo que se conoce como el ``Orden de las Operaciones''.  Una operación es cuando se usa un operador (uno de los símbolos de la tabla anterior).  Hay más operadores que esos símbolos básicos, pero para esa lista (suma, resta, multiplicación y división), es suficiente con saber que la multiplicación y la división tiene mayor orden de prioridad que la suma y la resta. Esto significa que en una fórmula, la parte de la multiplicación o la división se calcula antes que la parte de la suma y resta.  En la siguiente fórmula, todos los operadores son sumas (+), en este caso los números se suman en el orden que aparecen de izquierda a derecha:

\begin{listing}
\begin{verbatim}
>>> print(5 + 30 + 20)
55
\end{verbatim}
\end{listing}

\noindent
De igual manera, en esta otra fórmula, hay únicamente operadores de sumas y resta, por lo que de nuevo Python considera cada número en el orden en que aparece:

\begin{listing}
\begin{verbatim}
>>> print(5 + 30 - 20)
15
\end{verbatim}
\end{listing}

\noindent
Pero en la siguiente fórmula, hay un operador de multiplicación, por lo que los números 30 y 20 se toman en primer lugar. Esta fórmula lo que está diciendo es ``multiplica 30 por 20, y luego suma 5 al resultado'' (se multiplica primero porque tiene mayor orden que la suma): 

\begin{listing}
\begin{verbatim}
>>> print(5 + 30 * 20)
605
\end{verbatim}
\end{listing}

\noindent
¿Pero qué es lo que sucede cuando añadimos paréntesis? La siguiente fórmula muestra el resultado:

\begin{listing}
\begin{verbatim}
>>> print((5 + 30) * 20)
700
\end{verbatim}
\end{listing}

\noindent
¿Porqué el número es diferente? Porque lo paréntesis controlan el orden de las operaciones. Con los paréntesis, Python sabe que tiene que calcular primero todos los operadores que están dentro de los paréntesis, y luego los de fuera. Por eso lo que significa esta fórmula es ``suma 5 y 30 y luego multiplica el resultado por 20''.
El uso de los paréntesis puede volverse más complejo. Pueden existir paréntesis dentro de paréntesis:

\begin{listing}
\begin{verbatim}
>>> print(((5 + 30) * 20) / 10)
70
\end{verbatim}
\end{listing}

\noindent
En este caso, Python calcula los paréntesis \textbf{más internos} primero, luego los exteriores, y al final el otro operador. Por eso esta fórmula significa ``suma 5 y 30, luego multiplica el resultado por 20, y finalmente divide el resultado entre 10''. El resultado sin los paréntesis sería ligeramente diferente:

\begin{listing}
\begin{verbatim}
>>> 5 + 30 * 20 / 10
65
\end{verbatim}
\end{listing}

En este caso se multiplica primero 30 por 20, luego el resultado se divide entre 10, finalmente se suma 5 para obtener el resultado final.

\emph{Recuerda que la multiplicación y la división siempre van antes que la suma y la resta, a menos que se utilicen paréntesis para controlar el orden de las operaciones.}

\section{No hay nada tan voluble como una variable}\index{variable}

Una `variable' es un término de programación que se utiliza para describir un sitio en el que almacenar cosas. Las `cosas' pueden ser números, o textos, o listas de números y textos---y toda clase de otros elementos demasiado numerosos para profundizar aquí. Por el momento, vamos a decir que una variable algo así como un buzón o caja.

\begin{center}
\includegraphics*[width=76mm]{girlbubble.eps}
\end{center}

De la misma forma en que puedes meter cosas (como una carta o un paquete) en un un buzón, puedes hacerlo con una variable, puedes meter cosas (números, textos, listas de números y textos, etc, etc, etc) en ella. Con esta idea de buzón es con la que funcionan muchos lenguajes de programación. Pero no todos.

En Python, las variables son ligeramente diferentes. En lugar de ser un buzón con cosas dentro, una variable es más parecida a una etiqueta que pegas por fuera de la caja o carta. Podemos despegar la etiqueta y pegarla en otra cosa, o incluso atar la etiqueta (con un trozo de cuerda, tal vez) a más de una cosa. En Python las variables se crean al darles un nombre, utilizando después un símbolo de igual (=), y luego diciéndole a Python a qué cosa queremos que apunte este nombre. Por ejemplo:

\begin{listing}
\begin{verbatim}
>>> fred = 100
\end{verbatim}
\end{listing}

Acabamos de crear una variable llamada `fred' y le dijimos que apuntase al número 100. Lo que estamos haciendo es decirle a Python que recuerde ese número porque queremos usarlo más tarde. Para descubrir a qué está apuntando una variable, lo único que tenemos que hacer es teclear `print' en la consola, seguido del nombre de la variable, y pulsar la tecla Intro. Por ejemplo:

\begin{listing}
\begin{verbatim}
>>> fred = 100
>>> print(fred)
100
\end{verbatim}
\end{listing}

También le podemos decir a Python que queremos que la variable \code{fred} apunte a otra cosa diferente:

\begin{listing}
\begin{verbatim}
>>> fred = 200
>>> print(fred)
200
\end{verbatim}
\end{listing}

\noindent
En la primera línea decimos que ahora queremos que fred apunte al número 200. Luego, en la segunda línea, le pedimos a Python que nos diga a qué está apuntando fred para comprobar que lo hemos cambiado (ya no apunta a 100). También podemos hacer que más de un nombre apunte a la misma cosa:

\begin{listing}
\begin{verbatim}
>>> fred = 200
>>> john = fred
>>> print(john)
200
\end{verbatim}
\end{listing}

En el código anterior estamos diciendo que queremos que el nombre (o etiqueta) \code{john} apunte a la misma cosa a la que apunta \code{fred}.
Desde luego, `fred' no es un nombre muy útil para una variable. No nos dice nada sobre para qué se usa. Un buzón se sabe para lo que se usa---para el correo. Pero una variable puede tener diferentes usos, y puede apuntar a muchas cosas diferentes entre sí, por lo que normalmente queremos que su nombre sea más informativo.
\par
Supón que iniciaste la consola de Python, tecleaste `fred = 200', y después te fuiste---pasaste 10 años escalando el Monte Everest, cruzando el desierto del Sahara, haciendo puenting desde un puente en Nueva Zelanda, y finalmente, navegando por el rio Amazonas---cuando vuelves a tu ordenador, ¿podrías acordarte de lo que el número 200 significaba (y para qué servía)?

\noindent
\emph{Yo no creo que pudiera.}

\noindent
En realidad, acabo de usarlo y no tengo ni idea qué significa `fred=200' (más allá de que sea un \emph{nombre} apuntando a un número \emph{200}).  Por eso después de 10 años, no tendrás absolutamente ninguna oportunidad de acordarte.
\par
¡Ajá! Pero si llamáramos a nuestra variable: \emph{numero\_de\_estudiantes}.

\begin{listing}
\begin{verbatim}
>>> numero_de_estudiantes = 200
\end{verbatim}
\end{listing}

Podemos hacer esto porque los nombres de las variables pueden formarse con letras, números y guiones bajos (\_)---aunque no pueden comenzar por un número.  Si vuelves después de 10 años, `numero\_de\_estudiantes' aún tendrá sentido para ti. Puedes teclear:

\begin{listing}
\begin{verbatim}
>>> print(numero_de_estudiantes)
200
\end{verbatim}
\end{listing}

\noindent
E inmediatamente sabrás que estamos hablando de 200 estudiantes.  No siempre es importante que los nombres de las variables tengan significado. Puedes usar cualquier cosa desde letras sueltas (como la `a') hasta largas frases; y en ocasiones, si vas a hacer algo rápido, un nombre simple y rápido para una variable es suficientemente útil.  Depende mucho de si vas a querer usar ese nombre de la variable más tarde y recordar por el nombre en qué estabas pensando cuando la tecleaste.

\begin{listing}
\begin{verbatim}
este_tambien_es_un_nombre_valido_de_variable_pero_no_muy_util
\end{verbatim}
\end{listing}

\section{Utilizando variables}\index{Variables}

Ya conocemos como crear una variable, ahora ¿Cómo la usamos?  ¿Recuerdas la fórmula que preparamos antes?  Aquella que nos servía para conocer cuánto dinero tendríamos al final de año, si ganabas 5 euros a la semana haciendo tareas, 30 euros a la semana repartiendo periódicos, y gastabas 10 euros por semana.  Por ahora tenemos: 

\begin{listing}
\begin{verbatim}
>>> print((5 + 30 - 10) * 52)
1300
\end{verbatim}
\end{listing}

\noindent
¿Qué pasaría si convertimos los tres primeros números en variables?  Intenta teclear lo siguiente:

\begin{listing}
\begin{verbatim}
>>> tareas = 5
>>> repartir_periodicos = 30
>>> gastos = 10
\end{verbatim}
\end{listing}

\noindent
Acabamos de crear unas variables llamadas `tareas', `repartir\_periodicos' y `gastos'. Ahora podemos volver a teclear la fórmula de la siguiente forma:

\begin{listing}
\begin{verbatim}
>>> print((tareas + repartir_periodicos - gastos) * 52)
1300
\end{verbatim}
\end{listing}

Lo que nos da exactamente la misma respuesta. Pero qué sucedería si eres capaz de conseguir 2 euros más por semana, por hacer tareas extra.  Solamente tienes que cambiar la variable `tareas' a 7, luego pulsar la tecla de flecha para arriba ($\uparrow$) en el teclado un varias veces, hasta que vuelva a aparecer la fórmula en el prompt, y pulsar la tecla Intro:

\begin{listing}
\begin{verbatim}
>>> tareas = 7
>>> print((tareas + repartir_periodicos - gastos) * 52)
1404
\end{verbatim}
\end{listing}

Así hay que teclear mucho menos para descubrir que a final de año vas a tener 1404 euros. Puedes probar a cambiar las otras variables, y luego pulsar la tecla de flecha hacia arriba para que se vuelva a ejecutar el cálculo y ver qué resultado se obtiene.

\begin{listing}
\begin{verbatim}
Si gastases el doble por semana:
>>> gastos = 20
>>> print((tareas + repartir_periodicos - gastos) * 52)
884
\end{verbatim}
\end{listing}

Solamente te quedarán 884 euros al final de año. Por ahora esto es un poco más util, pero no mucho. No hemos llegado a nada realmente útil aún. Pero por el momento, es suficiente para comprender que las variables sirven para almacenar cosas.

\noindent
\emph{¡Piensa en un buzón con una etiqueta en él!}

\section{¿Un trozo de texto?}\index{strings}

Si has estado prestando atención, y no únicamente leyendo por encima el texto, recordarás que he mencionado que las variables se pueden utilizar para varias cosas---no únicamente para los números. En programación, la mayor parte del tiempo llamamos a los textos `cadenas de caracteres'. Esto puede parecer un poco extraño; pero si piensas que un texto es como un `encadenamiento de letras' (poner juntas las letras), entonces quizá tenga un poco más de sentido para ti.

\noindent
\emph{Si bueno, quizás no tenga tanto sentido.}

En cualquier caso, lo que tienes que saber es que una cadena es solamente un puñado de letras y números y otros símbolos que se juntan de forma que signifique algo. Todas las letras, números y símbolos de este libro podrían considerarse que forman una cadena. Tu nombre podría considerarse una cadena. Como podría serlo la dirección de tu casa.  El primer programa Python que creamos en el Capítulo \ref{ch:notallsnakeswillsquishyou}, utilizaba una cadena: `Hola mundo'.
\par
En Python, creamos una cadena poniendo comillas alrededor del texto. Ahora podemos tomar nuestra, gasta ahora inútil, variable \code{fred}, y asignarle una cadena de la siguiente forma:

\begin{listing}
\begin{verbatim}
>>> fred = "esto es una cadena"
\end{verbatim}
\end{listing}

\noindent
Ahora podemos mirar lo que contiene la variable \code{fred}, tecleando \code{print(fred)}:

\begin{listing}
\begin{verbatim}
>>> print(fred)
esto es una cadena
\end{verbatim}
\end{listing}

\noindent
También podemos utilizar comillas simples para crear una cadena:

\begin{listing}
\begin{verbatim}
>>> fred = 'esto es otra cadena'
>>> print(fred)
esto es otra cadena
\end{verbatim}
\end{listing}

Sin embargo, si intentas teclear más de una línea de texto en tu cadena utilizando comillas simple (') o comillas dobles ("), verás un mensaje de error en la consola. Por ejemplo, teclea la siguiente línea y pulsa Intro, y saldrá en pantalla un mensaje de error similar a esto:

\begin{listing}
\begin{verbatim}
>>> fred = "esta cadena tiene dos
  File "<stdin>", line 1
    fred = "esta cadena tiene dos
                      ^
SyntaxError: EOL while scanning string literal
\end{verbatim}
\end{listing}

\index{multi-line string}Hablaremos de los errores más tarde, pero por el momento, si quieres escribir más de una línea de texto, recuerda que tienes que usar 3 comillas simples:

\begin{listing}
\begin{verbatim}
>>> fred = '''esta cadena tiene dos
... líneas de texto'''
\end{verbatim}
\end{listing}

\noindent
Imprime el contenido de la variable para ver si ha funcionado:

\begin{listing}
\begin{verbatim}
>>> print(fred)
esta cadena tiene dos
líneas de texto
\end{verbatim}
\end{listing}

Por cierto, verás que salen 3 puntos (...) siempre que tecleas algo que continúa en otra línea (como sucede en una cadena que ocupa más de una línea). De hecho, lo verás más veces según avanzemos en el libro.

\section{Trucos para las cadenas}\label{trickswithstrings}

He aquí una interesante pregunta: ¿Cuánto es 10 * 5 (10 veces 5)? La respuesta es, por supuesto, 50.

\noindent
\emph{De acuerdo, para nada es una pregunta interesante.}

Pero ¿Cuánto es 10 * 'a' (10 veces la letra a)? Podría parecer una pregunta sin sentido, pero hay una respuesta para ella en el Mundo de Python:

\begin{listing}
\begin{verbatim}
>>> print(10 * 'a')
aaaaaaaaaa
\end{verbatim}
\end{listing}

También funciona con cadenas de más de un carácter:

\begin{listing}
\begin{verbatim}
>>> print(20 * 'abcd')
abcdabcdabcdabcdabcdabcdabcdabcdabcdabcdabcdabcdabcdabcdabcdabcdabcdabcdabcdabcd
\end{verbatim}
\end{listing}

Otro truco con cadenas consiste en incrustar valores.  Para ello puedes usar\%s, que funciona como marcador o espacio reservado para el valor que quieras incluir en la cadena. Es más fácil de explicar con un ejemplo:

\begin{listing}
\begin{verbatim}
>>> mitexto = 'Tengo %s años'
>>> print(mitexto % 12)
Tengo 12 años
\end{verbatim}
\end{listing}

En la primera línea, creamos la variable mitexto con una cadena que contiene algunas palabras y un marcador (\%s). El \%s es una especie de señal que le dice ``sustitúyeme con algo'' a la consola de Python. Así que en la siguiente línea, cuando ejecutamos \code{print(mytext)}, usamos el símbolo \%, para decirle a Python que reemplace el marcador con el número 12. Podemos reutilizar esa cadena y pasarle diferentes valores:

\begin{listing}
\begin{verbatim}
>>> mitexto = 'Hola %s, ¿Cómo estás hoy?'
>>> nombre1 = 'Joe'
>>> nombre2 = 'Jane'
>>> print(mitexto % nombre1)
Hola Joe, ¿Cómo estás hoy?
>>> print(mitexto % nombre2)
Hola Jane, ¿Cómo estás hoy?
\end{verbatim}
\end{listing}

En el ejemplo anterior, hemos creado 3 variables (mitexto, nombre1 y nombre2)---la primera almacena la cadena con el marcador. Por ello, podemos imprimir la variable `mitexto', y utilizando el operador \% pasarle el valor almacenado en las variables `nombre1' y `nombre2'.  Puedes usar más de un marcador:

\begin{listing}
\begin{verbatim}
>>> mitexto = 'Hola %s y %s, ¿Cómo estáis hoy?'
>>> print(mitexto % (nombre1, nombre2))
Hola Joe y Jane, ¿Cómo estáis hoy?
\end{verbatim}
\end{listing}

Cuando utilizas más de un marcador, necesitas que los valores que se usan para reemplazar las marcas se encuentren entre paréntesis---Así que el modo correcto de pasar 2 variables es (nombre1, nombre2). En Python a un conjunto de valores rodeados de paréntesis se le llama \emph{tupla}, y es algo parecido a una lista, de las que hablaremos a continuación.

\section{No es la lista de la compra}\index{lists}

Eggs, milk, cheese, celery, peanut butter, and baking soda.  Which is not quite a full shopping list, but good enough for our purposes. If you wanted to store this in a variable you could create a string:

\begin{listing}
\begin{verbatim}
>>> shopping_list = 'eggs, milk, cheese, celery, peanut butter, baking soda'
>>> print(shopping_list)
eggs, milk, cheese, celery, peanut butter, baking soda
\end{verbatim}
\end{listing}

Another way would be to create a `list', which is a special kind of object in Python:

\begin{listing}
\begin{verbatim}
>>> shopping_list = [ 'eggs', 'milk', 'cheese', 'celery', 'peanut butter', 
... 'baking soda' ]
>>> print(shopping_list)
['eggs', 'milk', 'cheese', 'celery', 'peanut butter', 'baking soda']
\end{verbatim}
\end{listing}

This is more typing, but it's also more useful.  We could print the 3rd item in the list by using its position (called its index position), inside square brackets []:

\begin{listing}
\begin{verbatim}
>>> print(shopping_list[2])
cheese
\end{verbatim}
\end{listing}

Lists start at index position 0---so the first item in a list is 0, the second is 1, the third is 2.  That doesn't make a lot of sense to most people, but it does to programmers.  Pretty soon, when you walk up some stairs you'll start counting with zero rather than one.  That will really confuse your little brother or sister.
\par
We can show all the items from the 3rd item up to the 5th in the list, by using a colon inside the square brackets:

\begin{listing}
\begin{verbatim}
>>> print(shopping_list[2:5])
['cheese', 'celery', 'peanut butter']
\end{verbatim}
\end{listing}

[2:5] is the same as saying that we are interested in items from index position 2 up to (but not including) index position 5.  And, of course, because we start counting with 0, the 3rd item in the list is actually number 2, and the 5th item is actually number 4. Lists can be used to store all sorts of items.  They can store numbers:

\begin{listing}
\begin{verbatim}
>>> mylist = [ 1, 2, 5, 10, 20 ]
\end{verbatim}
\end{listing}

\noindent
And strings:

\begin{listing}
\begin{verbatim}
>>> mylist = [ 'a', 'bbb', 'ccccccc', 'ddddddddd' ]
\end{verbatim}
\end{listing}

\noindent
And mixtures of numbers and strings:

\begin{listing}
\begin{verbatim}
>>> mylist = [1, 2, 'a', 'bbb']
>>> print(mylist)
[1, 2, 'a', 'bbb']
\end{verbatim}
\end{listing}

\noindent
And even lists of lists:

\begin{listing}
\begin{verbatim}
>>> list1 = [ 'a', 'b', 'c' ]
>>> list2 = [ 1, 2, 3 ]
>>> mylist = [ list1, list2 ]
>>> print(mylist)
[['a', 'b', 'c'], [1, 2, 3]]
\end{verbatim}
\end{listing}

In the above example, a variable called `list1' is created with 3 letters, `list2' is created with a 3 numbers, and `mylist' is created using list1 and list2. Things can get rather confusing, rather quickly, if you start creating lists of lists of lists of lists$\ldots$ but luckily there's not usually much need for making things that complicated in Python. Still it is handy to know that you can store all sorts of items in a Python list.

\noindent
\emph{And not just your shopping.}

\subsection*{\color{BrickRed}Replacing items}\index{lists!replacing}

We can replace an item in the list, by setting its value in a similar way to setting the value of a normal variable. For example, we could change celery to lettuce by setting the value in index position 3:

\begin{listing}
\begin{verbatim}
>>> shopping_list[3] = 'lettuce'
>>> print(shopping_list)
['eggs', 'milk', 'cheese', 'lettuce', 'peanut butter', 'baking soda']
\end{verbatim}
\end{listing}

\subsection*{\color{BrickRed}Adding more items...}\index{lists!appending}

We can add items to a list by using a method called `append'.  A method is an action or command that tells Python that we want to do something.  We'll talk more about methods later, but for the moment, to add an item to our shopping list, we can do the following:

\begin{listing}
\begin{verbatim}
>>> shopping_list.append('chocolate bar')
>>> print(shopping_list)
['eggs', 'milk', 'cheese', 'lettuce', 'peanut butter', 'baking soda', 
'chocolate bar']
\end{verbatim}
\end{listing}

Which, if nothing else, is certainly an improved shopping list.

\subsection*{\color{BrickRed}$\ldots$and removing items}\index{lists!removing}

We can remove items from a list by using the command `del' (short for delete).  For example, to remove the 6th item in the list (baking soda):

\begin{listing}
\begin{verbatim}
>>> del shopping_list[5]
>>> print(shopping_list)
['eggs', 'milk', 'cheese', 'lettuce', 'peanut butter', 'chocolate bar']
\end{verbatim}
\end{listing}

Remember that positions start at zero, so shopping\_list[5] actually refers to the 6th item.

\subsection*{\color{BrickRed}2 lists are better than 1}\index{lists!joining}

We can join lists together by adding them, as if we were adding two numbers:

\begin{listing}
\begin{verbatim}
>>> list1 = [ 1, 2, 3 ]
>>> list2 = [ 4, 5, 6 ]
>>> print(list1 + list2)
[1, 2, 3, 4, 5, 6]
\end{verbatim}
\end{listing}

\noindent
We can also add the two lists and set the result to another variable:

\begin{listing}
\begin{verbatim}
>>> list1 = [ 1, 2, 3 ]
>>> list2 = [ 4, 5, 6 ]
>>> list3 = list1 + list2
>>> print(list3)
[1, 2, 3, 4, 5, 6]
\end{verbatim}
\end{listing}

\noindent
And you can multiply a list in the same way we multiplied a string:

\begin{listing}
\begin{verbatim}
>>> list1 = [ 1, 2 ]
>>> print(list1 * 5)
[1, 2, 1, 2, 1, 2, 1, 2, 1, 2]
\end{verbatim}
\end{listing}

\noindent
In the above example, multiplying list1 by five is another way of saying ``repeat list1 five times''. However, division (/) and subtraction (-) don't make sense when working with lists, so you'll get errors when trying the following examples:

\begin{listing}
\begin{verbatim}
>>> list1 / 20
Traceback (most recent call last):
  File "<stdin>", line 1, in <module>
TypeError: unsupported operand type(s) for /: 'list' and 'int'
\end{verbatim}
\end{listing}

\noindent
or:

\begin{listing}
\begin{verbatim}
>>> list1 - 20
Traceback (most recent call last):
  File "<stdin>", line 1, in <module>
TypeError: unsupported operand type(s) for -: 'type' and 'int'
\end{verbatim}
\end{listing}

\noindent
You'll get a rather nasty error message.

\section{Tuples and Lists}\label{tuplesandlists}\index{tuples}

A tuple (mentioned earlier) is a little bit like a list, but rather than using square brackets, you use round brackets---e.g. `(' and `)'.  You can use tuples in a similar way to a list:

\begin{listing}
\begin{verbatim}
>>> t = (1, 2, 3)
>>> print(t[1])
2
\end{verbatim}
\end{listing}

The main difference is that, unlike lists, tuples can't change, once you've created them.  So if you try to replace a value like we did earlier with the list, you'll get another error message:

\begin{listing}
\begin{verbatim}
>>> t[0] = 4
Traceback (most recent call last):
  File "<stdin>", line 1, in ?
TypeError: 'tuple' object does not support item assignment
\end{verbatim}
\end{listing}

That doesn't mean you can't change the variable containing the tuple to something else.  For example, this code will work fine:

\begin{listing}
\begin{verbatim}
>>> myvar = (1, 2, 3)
>>> myvar = [ 'a', 'list', 'of', 'strings' ]
\end{verbatim}
\end{listing}

First we create the variable \code{myvar} pointing to a tuple of 3 numbers.  Then we change \code{myvar} to point at a list of strings. This might be a bit confusing at first.  But think of it like lockers in a school.  Each locker has a name tag on it. You put something in the locker, close the door, lock it, then throw away the key.  You then peel the name tag off, wander over to another empty locker, and stick something else in that (but this time you keep the key).  A tuple is like the locked locker.  You can't change what's inside it.  But you can take the label off and stick it on an unlocked locker, and then put stuff inside that locker and take stuff out---that's the list.

\section{Things to try}

\emph{In this chapter we saw how to calculate simple mathematical equations using the Python console.  We also saw how brackets can change the result of an equation, by controlling the order that operators are used.  We found out how to tell Python to remember values for later use---using variables---plus how Python uses `strings' for storing text, and lists and tuples, for handling more than one item.}
\par

\subsection*{Exercise 1}
Make a list of your favourite toys and name it \code{toys}.  Make a list of your favourite foods and name it \code{foods}.  Join these two lists and name the result \code{favourites}.  Finally print the variable \code{favourites}.

\subsection*{Exercise 2}
If you have 3 boxes containing 25 chocolates, and 10 bags containing 32 sweets, how many sweets and chocolates do you have in total?  (Note: you can do this with one equation with the Python console)

\subsection*{Exercise 3}
Create variables for your first and last name. Now create a string and use placeholders to add your name.


\newpage
