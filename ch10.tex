% ch10.tex
% This work is licensed under the Creative Commons Attribution-Noncommercial-Share Alike 3.0 New Zealand License.
% To view a copy of this license, visit http://creativecommons.org/licenses/by-nc-sa/3.0/nz
% or send a letter to Creative Commons, 171 Second Street, Suite 300, San Francisco, California, 94105, USA.


\chapter{Where to go from here}

Congratulations! You've made it to the end.
\par
What you've hopefully learned from this book, are basic concepts that will make learning other programming languages much simpler.  While Python is a brilliant programming language, one language is not \emph{always} the best tool for every task.  So don't be afraid of looking at other ways to program your computer, if it interests you.

For example, if you're interested in games programming, you can perhaps look at something like BlitzBasic (\href{http://www.blitzbasic.com}{www.blitzbasic.com}), which uses the Basic programming language. Or perhaps Flash (which is used by many websites for animation and games---for example, the Nickelodeon website, \href{http://www.nick.com}{www.nick.com}, uses a lot of Flash).

If you're interested in programming Flash games, possibly a good place to start would be `Beginning Flash Games Programming for Dummies', a book written by Andy Harris, or a more advanced reference such as `The Flash 8 Game Developing Handbook' by Serge Melnikov.  Searching for `flash games' on \href{http://www.amazon.com}{www.amazon.com} will find a number of books on this subject.

Some other games programming books are: `Beginner's Guide to DarkBASIC Game Programming' by Jonathon S Harbour (also using the Basic programming language), and `Game Programming for Teens' by Maneesh Sethi (using BlitzBasic). Be aware that BlitzBasic, DarkBasic and Flash (at least the development tools) all cost money (unlike Python), so Mum or Dad will have to get involved before you can even get started.

If you want to stick to Python for games programming, a couple of places to look are: \href{http://www.pygame.org}{www.pygame.org}, and the book `Game Programming With Python' by Sean Riley.

If you're not specifically interested in games programming, but do want to learn more about Python (more advanced programming topics), then take a look at `Dive into Python' by Mark Pilgrim (\href{http://www.diveintopython.org}{www.diveintopython.org}).  There's also a free tutorial for Python available at: \href{http://docs.python.org/tut/tut.html}{http://docs.python.org/tut/tut.html}.  There's a whole pile of topics we haven't covered in this basic introduction so, at least from the Python perspective, there's still a lot for you to learn and play with.
\par\par\noindent
\emph{Good luck and enjoy your programming efforts.}

\newpage