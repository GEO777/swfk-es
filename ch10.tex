% ch10.tex
% This work is licensed under the Creative Commons Attribution-Noncommercial-Share Alike 3.0 New Zealand License.
% To view a copy of this license, visit http://creativecommons.org/licenses/by-nc-sa/3.0/nz
% or send a letter to Creative Commons, 171 Second Street, Suite 300, San Francisco, California, 94105, USA.


\chapter{Cómo seguir desde aquí}

¡Felicidades! Has llegado hasta el final.
\par
Lo que has aprendido con este libro son conceptos básicos que te ayudarán a aprender otros lenguajes de programación.  Aunque Python es un estupendo lenguaje de programación, un solo lenguaje no es el mejor para todo. Pero no te preocupes si tienes que mirar otros modos de programar en tu ordenador, si te interesa el tema.

Por ejemplo, si estás interesado en programación de juegos posiblemente te interese mirar el lenguaje BlitzBasic (\href{http://www.blitzbasic.com}{www.blitzbasic.com}), que utiliza el lenguaje de programación Basic. O tal vez quieras mirar Flash (que es lo que utilizan muchas páginas web para hacer animaciones y juegos---por ejemplo la página web de Nickelodeon o la de tamagotchi utilizan mucho código escrito en Flash).

Si estás interesado en programar juegos Flash, un buen lugar para comenzar podría ser `Beginning Flash Games Programming for Dummies', un libro escrito por Andy Harris, o una referencia más avanzada como el libro `The Flash 8 Game Developing HandBook' por Serge Melnikov.  Si buscas en la página web de amazon \href{http://www.amazon.com}{www.amazon.com} por las palabras `flash games' encontrarás un buen número de libros sobre programación de juegos en Flash.

Algunos otros libros de programación de juegos son: `Beginners Guide to DarkBASIC Game Programming' por Jonathon S Harbour (que también utiliza el lenguaje de programación Basic), y `Game Programming for Teens' por Maneesh Sethi (que utiliza BlitzBasic). Que sepas que las herramientas de desarrollo de BlitzBasic, DarkBasic y Flash cuestan dinero (no como Python), así que mamá o papá tendrán que comprometerse a gastarse dinero antes de que puedas comenzar.

Si quieres seguir con Python para programar juegos, hay un par de sitios en los que puedes mirar: \href{http://www.pygame.org}{www.pygame.org}, y el libro `Game Programming With Python' por Sean Riley.

Si no estas interesado específicamente en programar juegos, pero quieres aprender más sobre Python (temas más avanzados), échale un vistazo a `Dive into Python' de Mark Pilgrim (\href{http://www.diveintopython.org}{www.diveintopython.org}).  Existe también un tutorial gratuito para Python que está disponible en: \href{http://docs.python.org/tut/tut.html}{http://docs.python.org/tut/tut.html}.  Hay muchos temas que no hemos abordado en esta introducción básica por lo que desde la perspectiva de Python, hay mucho que puedes aprender y practicar aún.
\par\par\noindent
\emph{Buena suerte y pásalo bien en tu aprendizaje sobre programación.}

\newpage
