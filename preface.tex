% preface.tex
% This work is licensed under the Creative Commons Attribution-Noncommercial-Share Alike 3.0 New Zealand License.
% To view a copy of this license, visit http://creativecommons.org/licenses/by-nc-sa/3.0/nz
% or send a letter to Creative Commons, 171 Second Street, Suite 300, San Francisco, California, 94105, USA.


\chapter*{Preface}\normalsize
    \addcontentsline{toc}{chapter}{Preface}
\begin{center}
{\em A Note to Parents...}
\end{center}
\pagestyle{plain}

\noindent
Dear Parental Unit or other Caregiver,

In order for your child to get started with programming, you're going to need to install Python (at least version 2.4 or greater) on your computer.
This is a fairly straight-forward task, but there are a few wrinkles depending upon what sort of Operating System you're using.  If you've just bought a shiny new computer, have no idea what to do with it, and that previous statement has filled you with a severe case of the cold chills, you'll probably want to find someone to do this for you.  Depending upon the state of your computer, and the speed of your internet connection, this could take anything from 15 minutes to a few hours.  See the instructions below for your specific platform...

\noindent
\emph{\color{BrickRed}Windows}

If you're using Windows, go to \href{http://www.python.org}{www.python.org} and download the latest Windows installer.  At time of writing, this is:
\begin{quote}
     \href{http://www.python.org/ftp/python/2.5/python-2.5.msi}{http://www.python.org/ftp/python/2.5/python-2.5.msi}
\end{quote}
Double-click the icon for the Windows installer (you do remember where you downloaded it to, don't you?), and then follow the instructions to install it in the default location (this is probably \emph{c:$\backslash$Python25}).

\noindent
\emph{\color{BrickRed}Mac OS X}

\noindent
If you're using Mac OS X, go to \href{www.pythonmac.org}{www.pythonmac.org} and download the Python package (as of Feb 2007, version 2.4.4).  If you're a Mac guru, you're probably already mumbling about the fact that Python is already installed on your system, or that Python 2.5 is available from python.org---both of which are true.  However, it's possible that neither the built-in version of Python nor version 2.5, downloadable from python.org, support Tkinter, which is a requirement for some of the examples in this book.  Save yourself the trouble and download the latest version of Python from this page:
\begin{quote}
    \href{http://pythonmac.org/packages/py24-fat/index.html}{http://pythonmac.org/packages/py24-fat/index.html}
\end{quote}
\noindent
Install by double-clicking on the .dmg file, and following the standard Mac install procedures.
If you still don't believe me, install the package you think should work, start up the Python console and try typing the following commands (one after the other) at the prompt: "import Tkinter" and "import turtle" (don't type the quotation marks).  If those commands succeed, then you were right and I was wrong.  I bow to your infinitesimally larger intelligence, Oh Great Mac Guru.

\noindent
\emph{\color{BrickRed}Linux}

\noindent
If you're a Linux user, download and install the latest version of Python (for example, 2.5) for your distribution.  Given the large number of Linux flavours, it's impossible to give exact details on installation for each---but chances are, if you're running Linux, you already know what you're doing anyway.  In fact, you're probably insulted by the very idea of being told how to install$\ldots$anything

\noindent
\emph{\color{BrickRed}After installation$\ldots$}

\noindent
$\ldots$You might need to sit down next to your child for the first few chapters, but hopefully after a few examples, they should be batting your hands away from the keyboard to do it themselves.  They should, at least, know how to use a text editor of some kind before they start (no, not a Word Processor, like Microsoft Word---a plain, old-fashioned text editor)---they should at least able to open and close files, create new text files and save what they're doing.  Apart from that, this book will try to teach the basics from there.
\\
\\
\noindent\\
Thanks for your time, and kind regards,
\noindent\\
THE BOOK